\documentclass[12pt]{article}
\usepackage{amsmath,amssymb,amsthm}
\newcommand{\RR}{\mathbb{R}}
\newcommand{\GL}{\text{GL}}
\newcommand{\inv}[1]{#1^{-1}}
\theoremstyle{definition}
\newtheorem*{fact}{Fact}
\newtheorem*{defn}{Definition}

\begin{document}

\author{Yakov Pechersky}

\title{Lecture 01}

\maketitle
\section{Lecture}
Required linear algebra to generate examples of groups.
\\

The set of \( n \times n \) matrices.
The notation used is \(i\)th row and \(j\)th column, so that
entry is \(a_{ij}\). The first entry is \(a_{11}\), then \(a_{12}\),
up to \(a_{1n}\). Last row is \(a_{n1}, \ldots, a_{nn}\).

Entries of a matrix are numbers, according to Artin. Before the theories of
rings or vector spaces or abstract fields, take the entries to be real numbers
\( a_{ij} \in \RR \). Call the collection of all such \( n \times n \) matrices
as \( M_n(\RR)\). That is a vector space over the real numbers, of dimension \(n^2\).

You can add matrices \(A, B\), with entries \((a_{ij}), (b_{ij})\), then
\(A + B = (a_{ij} + b_{ij})\), adding the elements in the \(i,j\)th place.
That's the addition law in the vector space of \(n \times n\) matrices.

You can also multiply by a real scalar \(\alpha\in\RR \) such that
\(\alpha\cdot A =(\alpha\cdot a_{ij})\).

This makes this a \( \RR \) vector space of \( \dim n^2 \). For some basis.

This would be true for \(m \times n\) matrices as well. But here, you can also
multiply them.
\\

Multiplication of \(n \times n\) matrices.

\begin{equation}
\begin{split}
 A \cdot B & = (c_{ij}) \\
 & = \sum_k a_{ik}b_{kj}
\end{split}
\end{equation}

What do matrices represent? Linear operators from
\(\RR^n \xrightarrow[T]{} \RR^n\),
where \(T\) is a linear transformation.
Then the multiplication of matrices is
the composition of transformations.

Then we have
\[\RR^n\xrightarrow[S]{\text{map by} B}\RR^n\xrightarrow[T]{\text{map by} A}\RR^n\]

This may not be equal to doing \(T\) and then \(S\). We do have 
that \(A + B = B + A\), but it is not necessarily true that
\(A \cdot B \neq B \cdot A\). Here is an example of not equality.

We have to use \(2 \times 2\) matrices, since multiplication of
\(1 \times 1\) matrices is just multiplication of real numbers, which is
commutative.

\[
\begin{bmatrix}
    0 & 1\\
    0 & 0
\end{bmatrix}
\begin{bmatrix}
    0 & 0\\
    0 & 1
\end{bmatrix} =
\begin{bmatrix}
    0 & 1\\
    0 & 0
\end{bmatrix}\]
\[
\begin{bmatrix}
    0 & 0\\
    0 & 1
\end{bmatrix}
\begin{bmatrix}
    0 & 1\\
    0 & 0
\end{bmatrix} =
\begin{bmatrix}
    0 & 0\\
    0 & 0
\end{bmatrix}
\] 

The zero matrix is the zero element of the vector space.
When you add it to any element, you get the same.
\\

Call the identity matrix 
\[I = \begin{pmatrix}
1 & & & & \mbox{\normalfont\Large\bfseries 0}\\
& 1 & & & \\
& & \ddots & & \\
& & & 1 & \\
\mbox{\normalfont\Large\bfseries 0} & & & & 1
\end{pmatrix}\]
which is distinguished that multiplying anything by \( I \),
you get the anything back
\[0 + A = A + 0 = A\]
\[A\cdot I = I \cdot A = A\]

There is also the distributive law
\[A(B+C) = AB + AC\]
and the associative law 
\[A(BC) = (AB)C\]

The normal laws of arithmetic except for the commutativity of multiplication.

Proving associativity is hideous if doing it by element.
\[\RR^n\xrightarrow[U]{\text{map by } C}\RR^n\xrightarrow[S]{\text{map by } B}\RR^n\xrightarrow[T]{\text{map by } A}\RR^n\]
By composition of transformation, associative law is obvious.
\\

A much trickier question about \(n \times n\) matrices is inversion.
We say \(A\) is invertible iff \(\exists \) a matrix \(B : AB = BA = I\).

This is tricky business, since not matrices are invertible. The zero matrix is never invertible.
\( 0\cdot A = A \cdot 0 = 0\). But \(I\) is invertible. Take \(B=I\).

For \(1 \times 1\) matrices \((a)\) is invertible iff \(a \neq 0\). \(B = (\frac{1}{a})\).

For \(2 \times 2\) matrices \(A = \left(\begin{smallmatrix}
    a & b\\
    c & d
\end{smallmatrix}\right)\) is invertible iff \(ad - bc \neq 0\).

Then, \(A^{-1}=\frac{1}{ad-bc}\left(\begin{smallmatrix}
    d & -b\\
    -c & a
\end{smallmatrix}\right)\).
\[
\begin{pmatrix}
    a & b\\
    c & d
\end{pmatrix}
\begin{pmatrix}
    d & -b\\
    -c & d
\end{pmatrix}
=
\begin{pmatrix}
    ad-bc & 0\\
    0 & ad-bc 
\end{pmatrix}
\]
In general, the question of whether a matrix is invertible
uses the determinant \(\det : M_n(\RR)\rightarrow\RR \).
For a \(2 \times 2\) matrix \(A\), \(\det A = ad-bc\).
For a general \(A\), \[\det(A) = \sum_{m! \text{ terms}}(\pm1)(\text{product of m matrix entries})\]
\begin{fact}
\(A\) is invertible \(\iff \det(A)\neq 0\).
\end{fact}
In this case, there is a formula for the inverse of \(A\).
That is, \(\exists_{\text{natural}} B: AB = BA = \det(A)\cdot I\).
But the formula for the determinant is about as useful as the formula
for multiplication of matrices using elements.

The determinant is the action of an operator on a single dimensional vector space constructed
from the multidimentional vector space.
\\

We're not interested in all of \(n \times n\) matrices, but a subset called

\(\GL_n(\RR)\subset M_n(\RR) = \{A: \det A\neq 0\} = \{A: \text{there is an unique inverse matrix }\inv{A}\} \)

For \(\GL_1\) matrices, everything but 0.

For \(\GL_2\) matrices, for example, not \(\left[\begin{smallmatrix}
    0 & 1\\
    0 & 0
\end{smallmatrix}\right]\), since the determinant is 0.

It is defined by the non-vanishing of a certain polynomial given by the determinant.

If the inverse exists, it is unique. There aren't two different matrices that serve as an inverse.

Suppose \(A\cdot B = A\cdot C = I\). But inverses are inverses on the left and the right.
Then \(B(AB) = B(AC)\). Use the associate law: \((BA)B = (BA)C\). Now, by inverses,
\(I\cdot B = I\cdot C\), so then \(B = C\).
\\

What are the properties of this subset? We gain and we lose by choosing this \(\GL_n\) subset. 
There is no addition defined on the set \(\GL_n(\RR)\).
This is true even for \(\GL_1\), like the sum of \(1 + (-1) = 0\).
So, not closed under addition or scalar multiplication by 0.

But it is closed under multiplication. Two proofs.

Suppose \(A, B\) are invertible. Show that \(A\cdot B\) is invertible. Find an inverse!
Use \(\inv{B}\cdot\inv{A}\). Then, \((\inv{B}\inv{A})(AB)=\inv{B}(\inv{A}A)B=\inv{B}\cdot I\cdot B=I\cdot\inv{B}\cdot B=I\cdot I=I\)
Take the determinant definition. The product should have nonzero determinant. We know that
\(\det(AB) = \det(A)\cdot\det(B)\), and the product of nonzero real numbers is a nonzero real number.

This set also contains the multiplicative identity \(I\).

Has multiplicative inverses \(\inv{A}\), by definition.

The product is associate.

This is the first and most important example of a group.
\\

A group \(G\) is a set with a product operation \(g,h\in G: g\cdot h\in G\), which is

1.\ associative \(g(hk) = (gh)k\)

2.\ has an identity element \(e\) or \(1\), \(eg=ge=g\)

3.\ has inverses \(\inv{g}: g\inv{g}=\inv{g}g=e\)

It does not have to be commutative. If \(gh=hg\) for all pairs, say \(G\) is commutative or Abelian.
Even the \(2 \times 2\) matrix aren't all commutative.

An Abelian group: \(\mathbb{Z} = \{0, \pm1, \pm2, \pm3, \ldots\} \),
with product as addition \(a + b\). We have \(e = 0, a+0=a, \inv{a}=-a\).

Any vector space \(V\) is a Abelian group. Just forget about scalar multiplication. Product is addition.

Start with any set \(T\). Let \(G = \{\text{all bijections } g:T\rightarrow T\}=\text{Sym}(T)\)
is a group under composition of transformations. The identity is the identity transformation.
The inverse is the inverse of a bijection. Composition is associative.

All groups arise on conditions of this invertible bijection.

How did \(\GL_n\) come about? We took all invertible maps of \(\RR^n\rightarrow\RR^n\) that
preserve the linearity of the vector space, and that is how we got invertible matrices.

Take \(T=\{1,\ldots,n\},\text{Sym}=S_n\), which is a finite group of order \(n!\).
Even at \(n=3\), which is 6 elements, it is not Abelian.

\section{Problem set}
\subsection{1.1.7}
\subsection{1.1.16}
\subsection{1.1.17}

\end{document}