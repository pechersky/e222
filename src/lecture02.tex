\documentclass[12pt]{article}
\usepackage{amsmath,amssymb,amsthm}
\newcommand{\RR}{\mathbb{R}}
\newcommand{\GL}{\text{GL}}
\newcommand{\inv}[1]{#1^{-1}}
\theoremstyle{definition}
\newtheorem*{fact}{Fact}
\newtheorem*{defn}{Definition}

\begin{document}

\author{Yakov Pechersky}

\title{Lecture 02}

\maketitle

\section{Lecture}

Previously: \(\GL_n(\RR) = \{\) all invertible \(n \times n\) matrices \(A\) with entries \(a_{ij} \in \RR\}\)

similarly: \(\GL_n(\mathbb{C}) = \{\) all invertible \(n \times n\) matrices \(A\) with entries \(a_{ij} \in \mathbb{C}\}\)
\(\GL_n(\mathbb{Q}) = \{\) all invertible \(n \times n\) matrices \(A\) with entries \(a_{ij} \in \mathbb{Q}\}\),
where \(\mathbb{Q}\) are the rational numbers.

All are groups \(G\).

A \emph{group} is \begin{itemize} \setlength\itemsep{1em}
    \item a set with a product structure \(a, b \in G, a \cdot b \in G\)
    \item associative \(a \cdot (b \cdot c) = (a \cdot b) \cdot c\)
    \item identity \(e\) such that \(a \cdot e = e \cdot a = a\)
    \item inverse \(a \mapsto \inv{a}\) such that \(a \cdot \inv{a} = \inv{a} \cdot a = e\)
\end{itemize}

\end{document}
